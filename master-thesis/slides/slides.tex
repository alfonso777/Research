%\documentclass[12pt,t]{beamer}
\documentclass[xcolor=svgnames]{beamer}

\useoutertheme{infolines}
\usetheme{Boadilla}
%\usetheme[heigth=7mm]{Rochester}
%\usetheme{Rochester}
%\usetheme{Warsaw}
\usecolortheme{whale}


\setbeamertemplate{blocks}[rounded][shadow=true]
\setbeamertemplate{navigation symbols}{}
\usepackage[brazil]{babel}
\usepackage[utf8]{inputenc}
\usepackage{graphicx}
\usepackage{subfigure} 					% uso de várias figuras numa só
\usepackage{algorithm}
\usepackage{algorithmic}
\usepackage{float}

\usepackage{color, colortbl}
%\usepackage[usenames,dvipsnames,svgnames,table]{xcolor}
\usepackage{xcolor}
\usepackage{booktabs}
\usepackage{comment}


%\floatname{algorithm}{Procedure}
\renewcommand{\algorithmicrequire}{\textbf{ \colorbox{yellow}{Input:} }}
\renewcommand{\algorithmicensure}{\textbf{  \colorbox{yellow}{Output:} } }
 \definecolor{keywordColor}{RGB}{165, 42, 42}
 \definecolor{componentColor}{RGB}{0, 0, 205}

\renewcommand{\algorithmicfor}{\textcolor[RGB]{165, 42, 42}{\textbf{For}}}
\renewcommand{\algorithmicendfor}{\textcolor[RGB]{165, 42, 42}{\textbf{End}}}
\renewcommand{\algorithmicdo}{\textcolor[RGB]{165, 42, 42}{\textbf{Do}}}
\renewcommand{\algorithmicreturn}{\textcolor[RGB]{165, 42, 42}{\textbf{Return}}}

\newcommand {\otoprule}{\midrule [\heavyrulewidth]}  % This is for getting a bold horizontal line in the



\bibliographystyle{apalike}
% ------numbering frames-----------------------------------------%
%\newcommand*\oldmacro{}%
%\let\oldmacro\insertshorttitle%
%\renewcommand*\insertshorttitle{%
%	  \oldmacro\hfill%
%	  \insertframenumber\,/\,\inserttotalframenumber
%}
% ---------------------------------------------------------------%

% ------ References ---------------------------------------------%
\usepackage[absolute,overlay]{textpos}
\newenvironment{reference}[2]{%
  \begin{textblock*}{\textwidth}(#1,#2)
      \footnotesize\it\bgroup\color{red!50!black}}{\egroup\end{textblock*}}
% ---------------------------------------------------------------%

\renewcommand{\figurename}{Figura}
\renewcommand{\tablename}{Tabela}

% -------------------------------------------------------------- %
% Pacote necessário para subfiguras
\usepackage{multimedia, multirow}
\graphicspath{{./figures/}}
% -------------------------------------------------------------- %



\title[Defesa de Mestrado]
    {Detecção de Violações de SLA em Coreografias de Serviços Web}

\author[V. A. Phocco-Diaz]
	{
		{\bf Candidato} \\
		Victoriano Alfonso Phocco Diaz \\%\footnote{O aluno recebeu apoio financeiro do CNPq, processo 133147/2009-6} \\ [1ex]
		{\bf Orientador} \\  Daniel Macêdo Batista 	\\
	}
\institute[IME-USP]
	{Instituto de Matemática e Estatística \\
	 Departamento de Ciência da Computação \\	
	 Universidade de São Paulo  \\  [1ex]
	 %\texttt{alfonso7@ime.usp.br}
	}

\date[Março 2013]{Março de 2013}
%\date[]{}

% logo of my university
\titlegraphic{
   \includegraphics[width=1.8cm, height=2.2cm]{usp_logo.png}\hspace*{4.75cm}~%
   \includegraphics[width=1.3cm, height=1.8cm]{logoIME_Texto.png}
}


%%% Slides ---------------------------------
\begin{document}
%\pgfdeclareimage[width=180pt,height=20pt]{test}{images/header.jpg}
%\logo{\pgfuseimage{test}}

% --- the titlepage frame ------------------------------------------%
\begin{frame}[plain]
  \titlepage
\end{frame}

\section*{Roteiro}
    \begin{frame}{Roteiro}
        \tableofcontents[subsectionstyle=hide]
    \end{frame}

% ------------------------------------------------------------------%
\AtBeginSection[]
{
	\begin{frame}<beamer>
		\frametitle{}
		\tableofcontents[currentsection,%, currentsubsection
		  sectionstyle=show/shaded,
		  %hideothersubsections
		  %subsectionstyle=hide, 
		  %hideallsubsections
		  %subsectionstyle=show/shaded, 
		]
	\end{frame}
}

\AtBeginSubsection[]{
  \frame{
  \tableofcontents[currentsection,currentsubsection,hideothersubsections,subsectionstyle=show/shaded ]
  }
}
% ------------------------------------------------------------------%
% ------------------------------------------------------------------%
% ---------------------- Problema    -------------------------------%
\section{Conceitos Básicos}

% -----  Serviços Web ----------%
%    \begin{frame}{Serviço Web}
%    	Definição pela W3C [W3C,2004]: %\cite{W3C2004}:
%    	\begin{block}{Serviço Web}\vspace{-.3\baselineskip}
%        	 \begin{quote}
%                 A Web service is a software system designed to support interoperable machine-to-machine interaction over a network. It has an
% interface described in a machine-processable format (specifically WSDL). Other systems interact with the Web service in a manner
%prescribed by its description using SOAP messages, typically conveyed using HTTP with an XML serialization in conjunction with other Web-related standards.
%        	 \end{quote}
%    	\end{block}
	%%Tecnologias
%     \tiny{
        %\begin{thebibliography}{1}
        %  \beamertemplatearticlebibitems
          %\bibitem[W3C,2004]{W3C2004}
           % Web Services Architecture Working Group {\em Web Service Architecture}.
            %W3C Working Group Note 11. 2004. \url{http://www.w3.org/TR/ws-arch/}. Último acesso em 07-09-2011.
            %\newblock {\em Bootstrapping Performance and Dependability Attributes of Web Services}.
            %\newblock IEEE International Conference on Web Services (ICWS'06). 2006:205-212.
       %\end{thebibliography}
%      }
%    \end{frame}

% -----  SOA ----------%
    \begin{frame}{SOA (1/2)}
            \begin{block}{SOA (Arquitetura Orientada a Serviços)}\vspace{-.3\baselineskip}
            É um estilo de arquitetura de software cujo princípio fundamental prega que as funcionalidades implementadas pelas
	    aplicações devem ser disponibilizadas na forma de serviços [SOA, 2006]. %\cite{SOA2006}.
            \end{block}

    \end{frame}

    \begin{frame}{SOA (2/2)}
         \only<1>{	
              \begin{figure}[!h]
                  \centering
                  \includegraphics[width=0.9\textwidth]{SOA-Triangle_1.png}
                  \caption{Triângulo da SOA (baseado em [W3C, 2002])}
              \end{figure}	
          }
          \only<2>{	
              \begin{figure}[!h]
                  \centering
                  \includegraphics[width=0.9\textwidth]{SOA-Triangle_2.png}
                  \caption{Triângulo da SOA (baseado em [W3C, 2002])}
              \end{figure}	
          }
          \only<3>{	
              \begin{figure}[!h]
                  \centering
                  \includegraphics[width=0.9\textwidth]{SOA-Triangle_3.png}
                  \caption{Triângulo da SOA (baseado em [W3C, 2002])}
              \end{figure}	
          }

          \only<4>{	
              \begin{figure}[!h]
                  \centering
                  \includegraphics[width=0.9\textwidth]{SOA-Triangle_4.png}
                  \caption{Triângulo da SOA (baseado em [W3C, 2002])}
              \end{figure}	
          }
          \only<5>{	
              \begin{figure}[!h]
                  \centering
                  \includegraphics[width=0.9\textwidth]{SOA-Triangle_5.png}
                  \caption{Triângulo da SOA (baseado em [W3C, 2002])}
              \end{figure}	
          }

    \end{frame}

% -----  SOC ----------%
    \begin{frame}{SOC}
       	\begin{block}{SOC (Computação Orientada a Serviços)}\vspace{-.3\baselineskip}
           É um novo paradigma de computação que utiliza serviços como blocos básicos de construção
           para suportar o desenvolvimento rápido, de baixo custo e de fácil composição de aplicações
           distribuídas heterogêneas [Papazoglou et al., 2006]. %\cite{Papazoglou2007}.
        \end{block}
        Elementos Chave:

        \begin{itemize}
           \item <1->Serviços.
           \item <1->SOA.
           \item <2->\colorbox{yellow}{Composição de Serviços}.
                \begin{itemize}
                  \item <3->\colorbox{yellow}{\textbf{Orquestração de Serviços}}.
                  \item <4->\colorbox{yellow}{\textbf{Coreografia de Serviços}}.
                \end{itemize}
           %\item <5->\colorbox{yellow}{QoS}.
        \end{itemize}

      %Adicionar padrões
    \end{frame}


% -----  Composição de Serviços ----------%
%    \begin{frame}{Composição de Serviços}
%        \begin{itemize}
%          \item \textbf{Serviço Composto}: Um serviço construído a partir de outros serviços. O serviço composto também é um serviço.
%          \item \textbf{Composição de Serviços}: Processo de  obter serviços compostos combinando e vinculando outros serviços.
%          \item <1->Abordagens:
%                \begin{itemize}
%                  \item <2->\colorbox{yellow}{\textbf{Orquestração de Serviços}}.
%                  \item <3->\colorbox{yellow}{\textbf{Coreografia de Serviços}}.
%                \end{itemize}
%        \end{itemize}
%    \end{frame}

    %----- Orquestração de Serviços -----%
    \begin{frame}{Orquestração de Serviços}
          \only<1>{	
              \begin{figure}[!h]
                  \centering
                  \includegraphics[width=0.25\textwidth]{Orchestration_1.png}
                  \caption{Orquestração de serviços}
              \end{figure}	
          }
          \only<2>{	
              \begin{figure}[!h]
                  \centering
                  \includegraphics[width=0.45\textwidth]{Orchestration_2.png}
                  \caption{Orquestração de serviços}
              \end{figure}	
          }
          \only<3>{	
              \begin{figure}[!h]
                  \centering
                  \includegraphics[width=0.5\textwidth]{Orchestration_3.png}
                  \caption{Orquestração de serviços}
              \end{figure}	
          }
         % \only<4>{	
         %     \begin{figure}[!h]
         %         \centering
         %         \includegraphics[width=0.5\textwidth]{Orchestration_4.png}
         %         \caption{Orquestração de serviços}
         %     \end{figure}	
         % }


    \end{frame}


    %----- Coreografia de Serviços -----%
    \begin{frame}{Coreografia de Serviços}
          \only<1>{	
              \begin{figure}[!h]
                  \centering
                  \includegraphics[width=0.5\textwidth]{ChoreographyA.png}
                  \caption{Coreografia de serviços}
              \end{figure}	
          }
          %\only<2>{	
           %   \begin{figure}[!h]
            %      \centering
            %      \includegraphics[width=0.6\textwidth]{ChoreographyB_1.png}
            %      \caption{Coreografia de serviços}
            %  \end{figure}	
          %}
          \only<2>{	
              \begin{figure}[!h]
                  \centering
                  \includegraphics[width=1.0\textwidth]{ChoreographyB_2.png}
                  \caption{Coreografia de serviços}
              \end{figure}	
          }
          \only<3>{	
              \begin{figure}[!h]
                  \centering
                  \includegraphics[width=1.0\textwidth]{ChoreographyB_3.png}
                  \caption{Coreografia de serviços}
              \end{figure}	
          }
    \end{frame}

    \begin{frame}{Coreografia de Processos}
      \begin{itemize}
	%\item <1-> A Choreography is specified in BPMN 2.0 (Process Choreography).
	  %\begin{itemize}
	  %  \item Differs in purpose and behavior from a standard BPMN Process (Process Orchestration).
	    %\item Formalizes the way business \colorbox{yellow}{Participants} \textbf{coordinate} their \colorbox{yellow}{interactions}.
	  % \item Formalizes the way business \textcolor{blue}{\textbf{Participants}} \textbf{coordinate} their \textcolor{blue}{\textbf{interactions}}.
	  %\end{itemize}
	%\item <2-> Focus on the exchange of information (\colorbox{yellow}{Messages}) between these Participants.
	%\item <1-> Focus on interactions through \textcolor{blue}{\textbf{messages exchanges}}.
	\item <1-> Uma Coreografia também é um processo.
	\item <2-> \textbf{BPMN} (Bussiness Process Model and Notation) é um padrão para modelagem de processo de negócios.
	\item <2-> \textbf{BPMN} suporta modelagem de coreografias.
	\item <3-> Duas abordagens de modelagem:
	  \begin{itemize}
	    \item <3-> \colorbox{yellow}{Modelo de Interconexão}%: With collaborations diagrams.
	    \item <4-> \colorbox{yellow}{Modelo de Interação}%:  BPMN Choreographies. using special activities (\textit{Choreography Activity}).
	  \end{itemize}
      \end{itemize}

    \end{frame}


  \begin{frame}{Modelo de Interconexão }
      \begin{itemize}
	\item Vistas públicas interconectadas.	
	\item Uso de atividades de processos comuns.
 	\item Colaborações em BPMN 2.
      \end{itemize}
   \begin{figure}[!h]
	    \centering
	    \includegraphics[width=0.6\textwidth]{figures/interconnection_choreography-pt.png}
	    %\caption{BPMN elements for modeling choreographies.}
    \end{figure}	
  \end{frame}

  \begin{frame}{ Modelo de Interação}
    \begin{itemize}
	  \item Interações \textbf{capturadas globalmente}.	
	  \item Blocos básicos de construção: \textbf{interações atômicas} entre participantes.
	  \item Suportado a partir do BPMN 2.
	\end{itemize}
    \begin{figure}[!h]
	      \centering
	      \includegraphics[width=0.7\textwidth]{figures/interaction_choreography2-br.png}
	      %\caption{BPMN elements for modeling choreographies.}
      \end{figure}	
  \end{frame}


    %----- Categorização de Coreografias -----%
    %\begin{frame}{Classificação das Linguagens de Coreografias}
    %  \begin{figure}[!h]
    %      \centering
    %      \includegraphics[width=0.7\textwidth]{ChoreographyCategorization.png}
    %      \caption{Classificação das linguagens de coreografias }
    %  \end{figure}	
    %\end{frame}


% ------------------------------------------------------------------%
% ------------------------------------------------------------------%
% -------------------- Objetivos------------------------------------%
\section{Problema}

    %----- Coreografia de Serviços -----%
    \begin{frame}{Problema a ser resolvido}
      \begin{figure}[!h]
	  \centering
	  \includegraphics[width=0.8\textwidth]{ChoreographySLAs.png}
	  \caption{Problema a ser resolvido}
      \end{figure}	
    \end{frame}

    \begin{frame}{Objetivos}
        \begin{block}{Objetivo principal}\vspace{-.3\baselineskip}
        	\begin{itemize}
                  %\item Propor uma técnica de monitoramento ``não intrusivo'' de coreografias de serviços Web usando SLAs.
        		  %\item Propor uma técnica para definir SLAs baseado em restrições probabilísticas de QoS.
                  \item Detectar violações de contratos probabilísticas de QoS em coreografias de serviços Web
            \end{itemize}
        \end{block}
        \begin{block}{Objetivos secundários}\vspace{-.3\baselineskip}
        	\begin{itemize}
        	      %\item Realizar a implementação do monitoramento ``não intrusivo''.
		  \item	Propor mecanismos para definir requisitos de QoS.
                  \item Propor uma técnica de monitoramento de coreografias de serviços Web com restrições de QoS
definidas probabilisticamente.
                  %\item Realizar avaliações da técnica de monitoramento.
		  \item Desenvolver um simulador de coreografias de serviços Web com suporte à definição de atributos de QoS
                  \item Realizar avaliações do mecanismo de monitoramento de coreografias. %técnica de monitoramento.
        	\end{itemize}
        \end{block}
    \end{frame}

\begin{comment}
    \begin{frame}{Justificativa}
    	\begin{itemize}
          \item <1->Importância da \textbf{coreografia} de serviços Web.
          \item <2->\textbf{QoS} é um fator importante na adaptação, seleção, otimização, composição na SOC. %\footnote{SOC: Computação Orientada a Serviços}.
          \item <3->\textbf{Monitoramento} é uma base para a reação (adaptação, reconfiguração, renegociação, etc).
              	\begin{itemize}
                  \item Detecção de falhas e violações de SLA. %\footnote{SLA: Acordo de Nível de Serviço}.
                \end{itemize}
          %\item <4->Monitoramento em tempo de execução Vs.  verificações e validações estáticas.
          \item <4->\textbf{Contratos probabilísticos} refletem melhor o comportamento dinâmico dos \textbf{atributos de QoS} dos serviços Web.
    	\end{itemize}
    \end{frame}
\end{comment}
    %----- Coreografia de Serviços -----%
%    \begin{frame}{Problema a ser resolvido}
%      \begin{figure}[!h]
%	  \centering
%	  \includegraphics[width=0.8\textwidth]{ChoreographySLAs.png}
%	  \caption{Problema a ser resolvido}
%      \end{figure}	
%    \end{frame}





% ------------------------------------------------------------------%
% ------------- Monitoramento baseado em QoS -----------------------%
% ------------------------------------------------------------------%
\section{QoS e Monitoramento em Coreografias de Serviços Web }

%\subsection{ QoS e SLA}
%----- Qualidade de Serviços -----
    \begin{frame}{Qualidade de Serviço}
        \begin{itemize}
          \item Qualidade de Serviço : QoS.
          \item \textbf{Funcionalidade/serviço} = Quais operações o sistema executa.
                \begin{itemize}
                    \item Exemplo: compra de passagens de avião.
                \end{itemize}

          \item \textbf{QoS/Característica Não Funcional} = Quão bem o sistema executa os serviços.
                \begin{itemize}
                    \item Exemplo: O tempo médio de resposta é 2 segundos.
                \end{itemize}
          \item Importante em Composição de Serviços : \textbf{\emph{QoS-aware Composition}}.
        \end{itemize}
  \end{frame}

%---------- Qualidade de Serviço ---------
    \begin{frame}{Qualidade de Serviço}
      \begin{figure}[!h]
          \centering
          \includegraphics[width=1.0\textwidth]{QoSTaxonomy.png}
          \caption{Taxonomia de atributos de QoS [Rosenberg et al.,2006] } %\cite{Rosenberg2006}}
          %\label{fig:QoST_SLA_Mapping_Transformation}
      \end{figure}	

      \tiny{
        %\begin{thebibliography}{4}
         %         \beamertemplatearticlebibitems
         %  \beamertemplatearticlebibitems
         % \bibitem[Rosenberg et al.,2006]{Rosenberg2006}
          %  Rosenberg F, Platzer C, Dustdar S. {\em Bootstrapping Performance and Dependability Attributes of Web Services}.
          %  IEEE International Conference on Web Services (ICWS’06). 2006:205-212.
            %\newblock {\em Bootstrapping Performance and Dependability Attributes of Web Services}.
            %\newblock IEEE International Conference on Web Services (ICWS’06). 2006:205-212.
       %\end{thebibliography}
      }
    \end{frame}

%---------- Cálculo de Qos ---------
    \begin{frame}{Cálculo de QoS}
      \only<1>{	
          \begin{figure}[!h]
              \centering
              \includegraphics[width=1.0\textwidth]{ServiceInvocationTimes_.png}
              \caption{Instantes de tempo na utilização de um serviço Web [Michlmayr et al.,2009]} %\cite{Michlmayr2009}}
              %\label{fig:QoST_SLA_Mapping_Transformation}
          \end{figure}	

          \tiny{
            %\begin{thebibliography}{5}

             % \bibitem[Michlmayr et al.,2009]{Michlmayr2009}
             %   Michlmayr A, Rosenberg F, Leitner P, Dustdar S. {\em Comprehensive QoS monitoring of Web services and event-based SLA violation detection}. Proceedings of the 4th International Workshop on Middleware for Service Oriented Computing - MWSOC  ’09. 2009:1-6.
            %\end{thebibliography}
          }
      }

%       \only<2>{	
%          \begin{figure}[!h]
%              \centering
%              \includegraphics[width=1.0\textwidth]{ServiceInvocationTimes_1.png}
%              \caption{Instantes de tempo na utilização de um serviço Web}
%              %\label{fig:QoST_SLA_Mapping_Transformation}
%          \end{figure}	
%      }

%      \only<3>{	
%          \begin{figure}[!h]
%              \centering
%              \includegraphics[width=1.0\textwidth]{ServiceInvocationTimes_2.png}
%              \caption{Instantes de tempo na utilização de um serviço Web}
%              %\label{fig:QoST_SLA_Mapping_Transformation}
%          \end{figure}	
%      }

 %      \only<4>{	
 %         \begin{figure}[!h]
 %             \centering
 %             \includegraphics[width=1.0\textwidth]{ServiceInvocationTimes_2b.png}
 %             \caption{Instantes de tempo na utilização de um serviço Web}
 %             %\label{fig:QoST_SLA_Mapping_Transformation}
 %         \end{figure}	
 %     }

      \only<2>{	
          \begin{figure}[!h]
              \centering
              \includegraphics[width=1.0\textwidth]{ServiceInvocationTimes_3.png}
              \caption{Instantes de tempo na utilização de um serviço Web}
              %\label{fig:QoST_SLA_Mapping_Transformation}
          \end{figure}	
      }

      \only<3>{	
          \begin{figure}[!h]
              \centering
              \includegraphics[width=1.0\textwidth]{ServiceInvocationTimes_5.png}
              \caption{Instantes de tempo na utilização de um serviço Web}
              %\label{fig:QoST_SLA_Mapping_Transformation}
          \end{figure}	
      }


      \only<4>{	
          \begin{figure}[!h]
              \centering
              \includegraphics[width=1.0\textwidth]{ServiceInvocationTimes_6.png}
              \caption{Instantes de tempo na utilização de um serviço Web}
              %\label{fig:QoST_SLA_Mapping_Transformation}
          \end{figure}	
      }

    \end{frame}

    %-------------SLA ---------------%
    \begin{frame}{SLA }
        \begin{itemize}
          \item <1-> \textbf{Contrato} = Acordo formal entre uma ou mais partes, define requisitos e garantias das partes.
          \item <2-> \textbf{SLA} = Contrato que envolve requisitos e garantias de QoS.
          \item <3-> \textbf{Um SLA consiste de}:
            \begin{itemize}
              \item Partes
              \item Operações do serviço:
                    \begin{itemize}
                      \item <4-> Operações
                      \item <5-> \textbf{Parâmetros de SLA}: define as métricas de QoS envolvidas.
                    \end{itemize}
              \item Obrigações:
                      \begin{itemize}
                        \item <6-> \textbf{Garantias de QoS (objetivos ou restrições).}
                        \item <7-> Ações a serem tomadas se as garantias forem descumpridas (\textbf{reação}).
                      \end{itemize}
            \end{itemize}
            %\item <7-> Padrões: \textbf{WSLA}, WS-Agreement, WSOL, entre outros.
        \end{itemize}
        %\begin{block}{}\vspace{-.3\baselineskip}
         %{\Large SLA (Acordo de Nível de Serviço)}
        %\end{block}
    \end{frame}

  %%SLA Example
\begin{comment}
   \begin{frame}{ Exemplo de SLA}
     \only<1>{	
          \begin{figure}[!h]
              \centering
              \includegraphics[width=0.5\textwidth]{SLAExample_1.png}
              \caption{Um exemplo simples de um SLA}
              %\label{fig:QoST_SLA_Mapping_Transformation}
          \end{figure}	
      }

      \only<2>{	
          \begin{figure}[!h]
              \centering
              \includegraphics[width=0.5\textwidth]{SLAExample_2.png}
              \caption{Um exemplo simples de um SLA}
              %\label{fig:QoST_SLA_Mapping_Transformation}
          \end{figure}	
      }
      \only<3>{	
          \begin{figure}[!h]
              \centering
              \includegraphics[width=0.5\textwidth]{SLAExample_3.png}
              \caption{Um exemplo simples de um SLA}
              %\label{fig:QoST_SLA_Mapping_Transformation}
          \end{figure}	
      }


   \end{frame}
\end{comment}

  %%QoS Agregation
   \begin{frame}{Agregação de QoS}
        \begin{itemize}
          \item Processo de obter o valor cumulativo da QoS da composição a partir dos valores de QoS dos seus serviços
          componentes.
          \item Não existe solução geral.
          \item \colorbox{yellow}{Depende do atributo de QoS e do modelo de composição.}
          \item Abordagens:
          \begin{itemize}
            \item Somas, Máximos, Mínimos, Médias, etc.
            \item Analíticas: Redes de Petri, Redes de Fila, etc.
            \item Heurísticas: Algoritmos Genéticos.
            \item \colorbox{yellow}{\textbf{Simulação}}.
          \end{itemize}

        \end{itemize}
    \end{frame}

%%QoS Aggregation example
\begin{comment}
    \begin{frame}{Exemplo de Agregação de QoS}
      \begin{figure}[!h]
          \centering
          \includegraphics[width=0.6\textwidth]{WorkflowPatterns-QoS.png}
          \caption{Exemplo de Agregação de QoS}
          %\label{fig:QoST_SLA_Mapping_Transformation}
      \end{figure}	
    \end{frame}
\end{comment}

% ------------------------------------------------------------------%
% ---------------- SLAs probabilísticos ----------------------------%
%\subsection{SLAs probabilísticos }
%--------------------------------------------------------
    \begin{frame}{Contratos Rígidos}
        \begin{itemize}
          \item <1-> Os contratos são tipicamente realizados em base a \textbf{restrições rígidas} (\emph{hard contracts}):
                \begin{itemize}
                  %\item <2->\colorbox{yellow}{Tempo de resposta $<$ 10 ms}
		  \item <2-> Tempo de resposta $<$ 10 ms.
                \end{itemize}
          \item <3->Contratos rígidos não refletem o comportamento dinâmico da QoS dos serviços Web.
        \end{itemize}
    \end{frame}


    \begin{frame}{Comportamento dinâmico de atributos de QoS}
      \begin{figure}[!h]
          \centering
          \includegraphics[width=0.6\textwidth]{hardContractsLimitations.png}
          \caption{Tempos de resposta de 20,000 chamadas de um serviço [Rosario et al., 2008] } %~\cite{Rosario2008}}
          %\label{fig:QoST_SLA_Mapping_Transformation}
      \end{figure}	

      \tiny{
        %\begin{thebibliography}{7}
         %         \beamertemplatearticlebibitems
         %   \bibitem[Rosario et al., 2008]{Rosario2008}
         %   Rosario S, Benveniste A, Haar S, Jard C. {\em Probabilistic QoS and Soft Contracts for Transaction-Based Web Services Orchestrations }. IEEE Transactions on Services Computing. 2008;1(4):187-200.
            %\newblock {\em {\tiny Probabilistic QoS and Soft Contracts for Transaction-Based Web Services Orchestrations }}.
            %\newblock {\tiny IEEE Transactions on Services Computing. 2008;1(4):187-200.}
        %\end{thebibliography}
      }
    \end{frame}

%---------------------------------------------------------------------------
    \begin{frame}{Contratos Não Rígidos}
        \begin{itemize}
          \item <1-> Contratos não rígidos ( \emph{soft contracts} ):
                \begin{itemize}
                  \item \textcolor{blue}{\textbf{Tempo de resposta $<$ 10 ms, em 95\% dos casos}}.
                \end{itemize}
                Desse jeito, não é possível compor esse tipo de restrições ou contratos, isto é,
                composição de restrições.
          \item <2-> \textbf{Solução:} contratos probabilísticos não rígidos (\emph{probabilistic soft contracts}).
                  \begin{itemize}
                  \item \textcolor{blue}{ \textbf{Para cada parâmetro de QoS (tempo de resposta). Eu ofereço sua distribuição de
                    probabilidade e garanto que não será pior do que isso}}.
                \end{itemize}
          \item <3-> As \textbf{restrições probabilísticas} podem ser compostas.
                \begin{itemize}
                  \item Existem algumas abordagens para orquestração.
                  \item \colorbox{yellow}{Não existem abordagens para coreografias}.
                  \item \colorbox{yellow}{Tratam somente tempo de resposta}.
                \end{itemize}
        \end{itemize}
    \end{frame}



%---- Definição de Contratos -------------------
%    \setbeamercovered{transparent}
%    \begin{frame}{Definição de Contratos}
%        \begin{itemize}
%          \item <1-> Na prática, as restrições ou contratos são definidos como um conjunto finito  de \textbf{quantis}  dos parâmetros de QoS.

%          \item <2-> Esses \textbf{quantis} definem uma distribuição empírica de probabilidade desses parâmetros de QoS.
%                \begin{itemize}
%                    \item <3->Por exemplo: %\textbf{quantis} de 25\%, 50\%, 90\%, 95\% e 98\% correspondem a tempos de %resposta máximos de 2.5ms, 4.5ms, 6.4ms, 13.8ms, e 23.5ms respectivamente.
%                        \begin{tabular}{|c|c|}
%                  \hline
                  % after \\: \hline or \cline{col1-col2} \cline{col3-col4} ...
%                  Quantis & Tempo de Resposta \\
%                  \hline
%                  25\% & 2.5 ms \\
%                  50\% & 4.5 ms \\
%                  90\% & 6.4 ms \\
%                  95\% & 13.8 ms \\
%                  98\% & 23.5 ms \\
%                  \hline
%                \end{tabular}
%                \end{itemize}

%          \item <4-> O conjunto de restrições ou contratos compõem um SLA.

          %\item <5->WSLA é o padrão para especificar SLAs.

%        \end{itemize}
%    \end{frame}



% ------------- Monitoramento baseado em QoS -----------------------%
% ------------------------------------------------------------------%
%\subsection{Monitoramento baseado em QoS }
    \begin{frame}{Monitoramento baseado em QoS}
	Responsabilidades:
        \begin{itemize}
          \item Mede e calcula valores de métricas de QoS, também inclui \textbf{ agregação} de valores dos atributos de QoS.
          \item Verifica se existe violação de alguma restrição de QoS.
          \item \colorbox{yellow}{ Monitoramento de Coreografias deve ser ``não intrusivo''}.
          %\item Outras responsabilidades.
        \end{itemize}
    \end{frame}

%----- Abordagens de Monitoramento  -----------------------------------

    \begin{frame}{Abordagens de Monitoramento }
  %      \only<1>{	
	  Monitoramento Não Intrusivo: \textbf{\textit{Probe-Request}}
          \begin{figure}[!h]
              \centering
              \includegraphics[width=0.8\textwidth]{MonitoringApproaches_3.png}
              \caption{Monitoramento mediante Probe-Request}
              %\label{fig:QoST_SLA_Mapping_Transformation}
          \end{figure}	
  %      }
    \end{frame}


\begin{comment}
  % ---- Monitoramento e Adaptação %
  \begin{frame} {Camadas no Monitoramento }
      \only<1>{	
          \begin{figure}[!h]
              \centering
              \includegraphics[width=0.8\textwidth]{MonitoringLayers_1.png}
              \caption{Camadas de Monitoramento}
          \end{figure}	
        }
      \only<2>{	
          \begin{figure}[!h]
              \centering
              \includegraphics[width=0.85\textwidth]{MonitoringLayers_2.png}
              \caption{Camadas de Monitoramento}
         \end{figure}	
        }
      \only<3>{	
          \begin{figure}[!h]
              \centering
              \includegraphics[width=0.8\textwidth]{MonitoringLayers_3.png}
              \caption{Camadas de Monitoramento}
          \end{figure}	
        }
      \only<4>{	
          \begin{figure}[!h]
              \centering
              \includegraphics[width=0.8\textwidth]{MonitoringLayers_4.png}
              \caption{Camadas do Monitoramento}
          \end{figure}	
        }
  \end{frame}


  %--  QoS e SLA Multi-camada%
  \begin{frame} {QoS multi-camada em coreografias de serviços Web}
          \begin{figure}[!h]
              \centering
              \includegraphics[width=0.65\textwidth]{./figures/Choreography-MultiTier.png}
              \caption{Integração de QoS e SLA multi-camada em coreografias (baseado em [[Rosenberg et al., 2009]]) } %(baseado em [[Rosenberg et al., 2009]])}
              %\label{fig:QoST_SLA_Mapping_Transformation}
          \end{figure}	
  \end{frame}
\end{comment}


% ------------------------------------------------------------------%
% ------------- Trabalhos Relacionados -----------------------%
% ------------------------------------------------------------------%
%\section{Trabalhos Relacionados}


% ------------------------------------------------------------------%
% -------------        PROPOSTA    ---------------------------------%
% ------------------------------------------------------------------%
\section{Mecanismos}
\subsection{Visão Geral}

    \begin{frame}{Arquitetura do monitoramento para detectar violações de SLA}
      \only<1>{	
          \begin{figure}
              \centering
              \includegraphics[width=0.9\textwidth]{MonitoringOverview_1.png}
              \caption{Perspectiva Geral do monitoramento proposto}
          \end{figure}	
      }

       \only<2>{	
          \begin{figure}
              \centering
              \includegraphics[width=0.9\textwidth]{MonitoringOverview_2.png}
              \caption{Perspectiva Geral do monitoramento proposto}
          \end{figure}	
      }
       \only<3>{	
          \begin{figure}
              \centering
              \includegraphics[width=0.9\textwidth]{MonitoringOverview_3.png}
              \caption{Perspectiva Geral do monitoramento proposto}
          \end{figure}	
      }

       \only<4>{	
          \begin{figure}
              \centering
              \includegraphics[width=0.9\textwidth]{MonitoringOverview_4.png}
              \caption{Perspectiva Geral do monitoramento proposto}
          \end{figure}	
      }

       \only<5>{	
          \begin{figure}
              \centering
              \includegraphics[width=0.9\textwidth]{MonitoringOverview_5.png}
              \caption{Perspectiva Geral do monitoramento proposto}
          \end{figure}	
      }

       \only<6>{	
          \begin{figure}
              \centering
              \includegraphics[width=0.80\textwidth]{MonitoringOverview_6.png}
              \caption{Perspectiva geral do monitoramento proposto}
          \end{figure}	
      }

      \only<7>{	
          \begin{figure}
              \centering
              \includegraphics[width=0.8\textwidth]{MonitoringOverview_7.png}
              \caption{Perspectiva geral do monitoramento proposto}
          \end{figure}	
      }
    \end{frame}


    \begin{frame}{Etapas}
      \only<1>{	
		  \begin{figure}
		  \centering
		  \includegraphics[width=1.0\textwidth]{MonitoringStages1.png}
		  \caption{Etapas para atingir o monitoramento de coreografias}
		  \end{figure}	
	  }
      \only<2>{	
		  \begin{figure}
		  \centering
		  \includegraphics[width=1.0\textwidth]{MonitoringStages2.png}
		  \caption{Etapas para atingir o monitoramento de coreografias}
		  \end{figure}	
	  }
      \only<3>{	
		  \begin{figure}
		  \centering
		  \includegraphics[width=1.0\textwidth]{MonitoringStages3.png}
		  \caption{Etapas para atingir o monitoramento de coreografias}
		  \end{figure}	
	  }	
    \end{frame}

   \begin{frame}{Contribuições}

      \begin{itemize}
          \item <1-> Mecanismos para \textbf{definir requisitos de QoS} em coreografias de serviços Web. Para tanto, utilizaram-se duas abordagens:
             \begin{itemize}
                  \item <2->\textbf{Analítica}:  Usando Redes de Petri Estocásticas Generalizadas (GSPN).
                  \item <3->\textbf{Simulações}: Por meio de avaliações de desempenho usando o simulador de coreografias.
	    \end{itemize}

         \item <4-> \textbf{Estabelecimento de contratos  probabilísticos} de QoS entre os serviços em   coreografias de serviços Web.
	 \item <5-> Técnicas de monitoramento de coreografias para \textbf{detectar violações de SLA} com restrições de QoS probabilísticas.
	 \item <6->  Desenvolvimento de um \textbf{simulador de coreografias} com suporte de \textbf{QoS}.
	 \item <7->\textbf{Atributos de QoS}: tempo de resposta, tempo de processamento, largura de banda e latência de rede.
       \end{itemize}

   \end{frame}



  \begin{frame}{Elementos do BPMN 2 Suportados }
      \begin{figure}[!h]
	\centering
	\includegraphics[width=0.9\textwidth]{./figures/BPMNBasicChoroegraphy.png}
	\caption{Elementos BPMN suportados.}
	\label{fig:ChoreographyElements}
    \end{figure}

  \end{frame}



  \begin{frame}{ Modelo de QoS  (I)}
    \begin{itemize} %%Perhaps it's needed use some picture, for example life cycle call service.
	%\item Os atributos de QoS envolvem: \textbf{service}, \textbf{network} and \textbf{message} aspects.
	\item Atributos de QoS:
	    \begin{itemize}
	      \item <2->\colorbox{yellow}{Serviço}: \textbf{tempo de processamento, tempo de resposta }.
	      \item <3->\colorbox{yellow}{Rede}: \textbf{largura de banda}, \textbf{atraso} e erros de comunicação.
	      \item <4->\colorbox{yellow}{Mensagem}: \textbf{integridade da mensagem}.
	    \end{itemize}
    \end{itemize}
  \end{frame}

  \begin{frame}{ Modelo de QoS  (II)}
    \only<1>{	
	\begin{figure}[h]
	    \centering
	    \includegraphics[width=.6\linewidth]{figures/ChorInteractionToServiceInteraction.png}
	    \caption{ Interação de serviços a partir de  interações atômicas do BPMN2.}
	    \label{figure:InteractionBPMNServiceInteraction}
	\end{figure}
    }
    \only<2>{	
	\begin{figure}[h]
	    \centering
	    \includegraphics[width=.6\linewidth]{figures/QoSInvocationService.png}
	    \caption{Atributos de QoS em uma interação com um serviço Web.}
	    \label{figure:QoSInvocationService}
	\end{figure}
    }
  \end{frame}


%%Formulas?

\subsection{Definição de requisitos de QoS}
  \begin{frame}{Definição de requisitos de QoS: Analiticamente}
      \begin{enumerate}
	\item <1-> \textbf{Mapeamento} de uma coreografia para uma GSPN ( Rede Petri Estocástica Generalizada).
	    %\begin{itemize}
	    %  \item The choreography is specified according to ``interaction model''.
	    %  \item The choreography is specified in BPMN 2.0.
	    %  \item The resulting GSPN include a QoS model.
	    %\end{itemize}
	\item <2-> \textbf{Configurações} da GSPN obtida de acordo com cenários.
	\item <3-> \textbf{Simulações} dos cenários.
      \end{enumerate}

  \end{frame}
  

  \begin{frame}{Mapeamento de BPMN para GSPN (I)}
      \only<1>{
	  \begin{figure}[!h]
	  \centering
	  \includegraphics[width=0.8\textwidth]{BPMNChoreographyElements2_a.png}
	  \caption{Mapeamento de eventos e \textit{gateways} para módulos GSPN.}
	\end{figure}
      }
      \only<2>{
	\begin{figure}[!h]
	  \centering
	  \includegraphics[width=0.8\textwidth]{BPMNChoreographyElements2_b.png}
	  \caption{Mapeamento de eventos e \textit{gateways} para módulos GSPN.}
	\end{figure}
      }
    
  \end{frame}


  \begin{frame}{Mapeamento de BPMN para GSPN (II)}
	\only<1>{	
	      \begin{figure}[!h]
		  \centering
		  \includegraphics[width=1.0\textwidth]{BPMNTaskChoreography-QoS_Model_a-br.png}
	      \end{figure}
	  }
	  \only<2>{	
	      \begin{figure}[!h]
		  \centering
		  \includegraphics[width=1.0\textwidth]{BPMNTaskChoreography-QoS_Model_b-br.png}
	      \end{figure}
	  }

	\only<3>{	
	      \begin{figure}[!h]
		  \centering
		  \includegraphics[width=1.0\textwidth]{BPMNTaskChoreography-QoS_Model_1-br.png}
	      \end{figure}
	  }
	\only<4>{	
	      \begin{figure}[!h]
		  \centering
		  \includegraphics[width=1.0\textwidth]{BPMNTaskChoreography-QoS_Model_2-br.png}
	      \end{figure}
	  }
      \only<5>{
	      \begin{figure}[!h]
		  \centering
		  \includegraphics[width=1.0\textwidth]{BPMNTaskChoreography-QoS_Model_3-br.png}
	      \end{figure}
      }
      \only<6>{
	      \begin{figure}[!h]
		  \centering
		  \includegraphics[width=1.0\textwidth]{BPMNTaskChoreography-QoS_Model_4-br.png}
	      \end{figure}
      }
      \only<7>{
	      \begin{figure}[!h]
		  \centering
		  \includegraphics[width=1.0\textwidth]{BPMNTaskChoreography-QoS_Model_5-br.png}
	      \end{figure}
      }
  \end{frame}


  \begin{frame}{Algoritmo de Mapeamento (I)}
      \begin{algorithm}[H]
	  %\caption{Mapping a Interconnection Choreography in BPMN onto a GSPN with QoS model}
	    %\caption{ { \small Mapeamento de uma coreografia especificada em BPMN 2.0 para uma GSPN com o modelo de QoS } }
	    \caption*{ Mapeamento de uma coreografia em BPMN 2.0 para uma GSPN com suporte de QoS}
	    %\label{alg2}

	    \begin{algorithmic}%[1]
	      { \footnotesize		
			    %\REQUIRE Process Choreography $\mathcal{PC}$ in BPMN 2.0.
			    \visible<1->{ \REQUIRE \textbf{Process Choreography} \textcolor{componentColor}{ \textbf{ $PC = (\mathcal{O, A, E, G, T}, \{e^S\}, \mathcal{E}^I,$ $\{e^E\}, \mathcal{E}^{I_M}, \mathcal{E}^{I_T},
			    \mathcal{G}^F, \mathcal{G}^J,$ $\mathcal{G}^X, \mathcal{G}^M, \mathcal{G}^V, \mathcal{F} )$} } in BPMN 2.0. }

			    \visible<2->{ \ENSURE Generalized Stochastic Petri Net \textcolor{componentColor}{ \textbf{$GSPN_{QoS}$} }.  }
			    \vspace*{3 mm}	
	      }	

	\end{algorithmic}

      \end{algorithm}
  \end{frame}


 \begin{frame}{ Algoritmo de Mapeamento (II)}
    \only<1>{	
        \begin{figure}[!h]
            \centering
            \includegraphics[width=0.5\textwidth]{Algorithm1.png}
        \end{figure}
    }
    \only<2>{	
        \begin{figure}[!h]
            \centering
            \includegraphics[width=0.5\textwidth]{Algorithm2.png}
        \end{figure}
    }
    \only<3>{	
        \begin{figure}[!h]
            \centering
            \includegraphics[width=0.65\textwidth]{Algorithm3.png}
        \end{figure}
    }
    \only<4>{	
        \begin{figure}[!h]
            \centering
            \includegraphics[width=0.65\textwidth]{Algorithm4.png}
        \end{figure}
    }
 \end{frame}




%\subsection{ChorSim: Simulador de Coreografias}
  \begin{frame}{Definição de Requisitos por meio de simulações}
     
     \only<1>{
	\framesubtitle{ChorSim: Simulador de Coreografias} 
	\begin{figure}[!h]
	  \centering
	  \includegraphics[width=.6\linewidth]{figures/ServiceDependency_Events_QoS.png}
	  \caption{Atributos de QoS calculados em um evento dado. (1) Recebendo requisições de um cliente ou serviço. (2) enviando
	  requisições para um outro serviço. (3) recebendo resposta de um outro serviço (dependência). (4) enviando resposta para um cliente ou serviço solicitador.}
	  \label{figure:QoSAttributosEvents}
	\end{figure}
    }
    \only<2>{
      \framesubtitle{Arquitetura do ChorSim} 
      \begin{figure}
	  \centering
	  \includegraphics[width=0.7\textwidth]{figures/Architecture-Simulator.png}
	  % MonitoringOverview.png: 1756x1462 pixel, 250dpi, 17.84x14.85 cm, bb=0 0 506 421
	  \caption{Arquitetura do simulador de coreografias.}
	  \label{fig:chorsim_architecture}
	\end{figure}
    }
  
  \end{frame}


%%TODO:  Mais imagens pro simulador, troços de código?



  \subsection{Estabelecimento do contrato probabilístico}

  \begin{frame}{Estabelecimento de SLAs Probabilísticos}

        \textcolor{Blue}{$$F_S(x) = P(\delta_S \leq x)$$ } é a função distribuição acumulada (fda) de um parâmetro
       de QoS \textcolor{Blue}{$\delta$} do serviço \textcolor{Blue}{$S$}.
      \pause

      \begin{enumerate}
      \item <1->\textbf{Condições Iniciais}:
	  \begin{itemize}
	    \item <1-> Configurar a plataforma e a implantação no ChorSim.
	    \item <2-> Definir e estimar a \textbf{fda} \textcolor{Blue}{$Fs_i$} para cada  serviço \textcolor{Blue}{$S_i$}.
	    %\item <3-> Definir os quantis que representem todo \textcolor{Blue}{$Fs_i$}.
	  \end{itemize}

      \item <3-> \textbf{Simulação usando \textbf{ChorSim}}:
	  \begin{enumerate}
	  \item <3-> \label{a} Para cada invocação de um serviço \textcolor{Blue}{$s_i$},  um valor \textcolor{Blue}{$q_i$} do parâmetro de QoS 
	    é obtido a partir da simulação em ChorSim.%de \textcolor{Blue}{$Fs_i(x)$}.
	  \item <4-> \label{b} \textbf{Agregação}: Estimar o valor \textcolor{Blue}{$Q$} do parâmetro de QoS do serviço composto \textcolor{Blue}{$S$} 
	  a partir dos valores obtidos 	  no passo anterior usando o ChorSim. %precisa de um modelo (ex.: modelo de execução de ordens parciais)

	  \item <5-> Rodar as simulações dos passos 2.1 e 2.2 várias vezes, o suficiente para estimar \textcolor{Blue}{$F_S$}
	  empiricamente a partir dos valores \textcolor{Blue}{$Q$} estimados .%end-to-end
	  \end{enumerate}

	\item <6-> A partir da \textbf{fda}  \textcolor{Blue}{$F_S$}  se selecionam quantis para definir o contrato.

     % simulação com o método de Monte-Carlo:
    \end{enumerate}
  \end{frame}

  

\subsection{Monitoramento de coreografias }


  % ---- Monitoramento Probabilístico de Coreografias%
  \begin{frame} { Monitoramento }

      \begin{itemize}
	\item \textcolor{Blue}{$F_S$}:  Função de distribuição acumulada do \textbf{contrato}.
	\item \textcolor{Blue}{$\Delta$}: Um conjunto finito de amostras dos valores de um parâmetro de QoS do serviço \textcolor{Blue}{$S$}.
	%\item \textcolor{Blue}{$f'_{si}$}: Distribuição de probabilidade empírica de um serviço.
	%\item Agregação de QoS.
	\item \textcolor{Blue}{$F'_S$}: Função de distribuição acumulada  estimada usando ChorSim.
	%\item \textcolor{Blue}{$F_s$} : Função Distribuição de Probabilidade parâmetro de QoS \textcolor{Blue}{$p$} do %serviço composto \textcolor{Blue}{$S$}.
	\item \textcolor{Blue}{$\lambda$}: Zona de tolerância.
      \end{itemize}


        \begin{equation}
            \textcolor{Blue}{
                F'_{S,\Delta}(x) = \frac{ | \{\delta, \delta \in \Delta \leq x  \}|}{|\Delta|}
            }
        \end{equation}
    	
	\pause
        %\begin{equation}
        %\textcolor{Blue}{
        %    \forall x \in R^{+} : F'_{s,\Delta}(x) \geq F_s(x)
        %     }
        %\end{equation}

        \begin{equation}
            \textcolor{Blue}{
                 \exists x \in R^{+} : F'_{s,\Delta}(x) < F_s(x)
            }
        \end{equation}

	\pause
	%\colorbox{yellow}{\parbox{0.5\textwidth}{
        \begin{equation}
            \textcolor{Blue}{
              sup_{x \in R^+}  (F'_{s,\Delta}(x) - F_s(x) ) \geq \lambda  \footnote{Rosario et al., 2009}
            }
        \end{equation}
	%}}
  \end{frame}


  \begin{frame}{ Detecção de Violações de SLA (I)}
      \textbf{Problema:} Dominância estocástica

      \begin{equation}	\textcolor{Blue}{
	\begin{align}
	  H_0 : & \quad \forall x, F_S(x) >=F'_S(x)   \notag\\
	  contra \notag\\
	  H_1 : &  \quad \exists x, F_S(x) < F'_S(x) 
	\end{align}}
      \end{equation}
      \pause 
    \textbf{Solução:} Teste de Kolmogorov-Smirnov de apenas um lado %de duas amostras%\textit{One-sided two-sample Kolmogorov-Smirnov test} 
     \only<2>{
	\begin{equation}	\textcolor{Blue}{
	    [D,p] = kstest(X_{contract}, X_{monitoring}, KS_{side})
	  }
	\end{equation}
      }
      \only<3>{
	  \begin{equation*} 	\textcolor{Blue}{
	    \begin{align}
	      [D^{+},p^{+}] & = kstest(X_{contract}, X_{monitoring}, greater) %\notag\\
	     % [D^{-},p^{-}] & = kstest(X_{contract}, X_{monitoring}, less)
	    \end{align} }
	  \end{equation*} 
      }
      \pause
      \begin{equation*}
	\textcolor{Blue}{
	  verify(X_{monitoring}) = 	
	  \begin{cases}
	  true,  & \text{ se } p^{+}>= \alpha \wedge D^{+}< \lambda \\
	  false, &  \text{ de outra maneira }
	  \end{cases}
	}
      \end{equation*}

  \end{frame}


\begin{comment}
%%TODO: precisa-se explica ro que é KS com graficos e tudo o mais?
  \begin{frame}{Detecção de Violações de SLA (II)}
    Então, para a detecção de violações usam-se:
      \begin{equation*} 	\textcolor{Blue}{
	\begin{align}
	  [D^{+},p^{+}] & = kstest(X_{contract}, X_{monitoring}, greater) \notag\\
	  [D^{-},p^{-}] & = kstest(X_{contract}, X_{monitoring}, less)
	\end{align} }
      \end{equation*} 

    \pause
    Portanto,  para que um conjunto de amostras \textcolor{Blue}{$X_{monitoring}$} cumpra o contrato uma regra baseada
    em \textcolor{Blue}{$p^{+}$} e \textcolor{Blue}{$D$} deve ser definida:

    \pause
      \begin{equation*}
	\textcolor{Blue}{
	  verify(X_{monitoring}) = 	
	  \begin{cases}
	  true,  & \text{ se } p^{+}>= \alpha \wedge D^{+}< \lambda \\
	  false, &  \text{ de outra maneira }
	  \end{cases}
	}
      \end{equation*}
    
  \end{frame}
\end{comment}



% ------------------------------------------------------------------%
% -------------    Experimentos e Resultados   ---------------------%
% ------------------------------------------------------------------%
\section{Experimentos e Resultados}
\subsection{ Definição de Requisitos de QoS Analiticamente}

  \begin{frame}{Cenário de Coreografia para a abordagem analítica}
    \begin{figure}[!h]
	\centering
	\includegraphics[width=0.7\textwidth]{figures/Example-InteractionChor-br.png}
	\caption{Exemplo de modelo de interação de coreografias, oferta de investimento. }%Interconnection model describing investment offers 
	\label{fig:Example-InteractionChor}
    \end{figure}
  \end{frame}

  \begin{frame}{Mapeamento}
    \begin{figure}[!h]
    	\centering
    	\includegraphics[width=1.0\textwidth]{BPMNChoreographyExample-QoS.png}
    	\caption{GSPN obtida após o mapeamento.}
    \end{figure}
  \end{frame}



  \begin{frame}{ Configuração }
      \only<1>{
	      \begin{figure}[!h]
		  \centering
		  \includegraphics[width=1.0\textwidth]{BPMNChoreographyExample-QoS_configuration1-br.png}
	      \end{figure}
      }
      \only<2>{
	      \begin{figure}[!h]
		  \centering
		  \includegraphics[width=1.0\textwidth]{BPMNChoreographyExample-QoS_configuration2-br.png}
	      \end{figure}
      }
      \only<3>{
	      \begin{figure}[!h]
		  \centering
		  \includegraphics[width=1.0\textwidth]{BPMNChoreographyExample-QoS_configuration3-br.png}
	      \end{figure}
      }
      \only<4>{
	      \begin{figure}[!h]
		  \centering
		  \includegraphics[width=1.0\textwidth]{BPMNChoreographyExample-QoS_configuration4-br.png}
	      \end{figure}
      }
      \only<5>{
	      \begin{figure}[!h]
		  \centering
		  \includegraphics[width=1.0\textwidth]{BPMNChoreographyExample-QoS_configuration5-br.png}
	      \end{figure}
      }
      \only<6>{
	      \begin{figure}[!h]
		  \centering
		  \includegraphics[width=1.0\textwidth]{BPMNChoreographyExample-QoS_configuration6-br.png}
	      \end{figure}
      }
      \only<7>{
	      \begin{figure}[!h]
		  \centering
		  \includegraphics[width=1.0\textwidth]{BPMNChoreographyExample-QoS_configuration7-br.png}
	      \end{figure}
      }
  \end{frame}


   \begin{frame}{ Simulações na GSPN }
      \begin{itemize}
	\item <1-> A ferramenta \textbf{Pipe2} foi usada para \textbf{modelar} e \textbf{simular} a \textbf{GSPN}.
	\item <2-> \textbf{$1$ token} = \textbf{$1$ instância de coreografia}.
	\item <2-> \textbf{$100$ tokens} são considerados para cada cenários na posição de inicio (\textbf{Start}).
	\item <2-> \textbf{$100$ instâncias concorrentes} foram executadas (\textit{multiple-server semantic}).
	\item <3-> $1500$ disparos e $10$ replicações.
	\item <3-> Nível de confiança de $95\%$.
      \end{itemize}
    \end{frame}

    \begin{frame}{ Resultados }
	\only<1>{
	  \begin{table} [tp] %[!h]
		\centering
		\caption{Resultados das simulações} % (in \%) }
		\begin{figure}[!h]
		    \centering
		    \includegraphics[width=0.9\textwidth]{results1-gspn.png}
		\end{figure}
	  \end{table}
	}
	\only<2>{
	  \begin{table} [tp] %[!h]
		\centering
		\caption{Resultados das simulações} % (in \%) }
		\begin{figure}[!h]
		    \centering
		    \includegraphics[width=0.9\textwidth]{results2-gspn.png}
		\end{figure}
	  \end{table}
	}
	\only<3>{
	  \begin{table} [tp] %[!h]
		\centering
		\caption{Resultados das simulações} % (in \%) }
		\begin{figure}[!h]
		    \centering
		    \includegraphics[width=0.9\textwidth]{results3-gspn.png}
		\end{figure}
	  \end{table}
	}
	\only<4>{
	  \begin{table} [tp] %[!h]
		\centering
		\caption{Resultados das simulações} % (in \%) }
		\begin{figure}[!h]
		    \centering
		    \includegraphics[width=0.9\textwidth]{results4-gspn.png}
		\end{figure}
	  \end{table}
	}
  \end{frame}


\subsection{Definição de requisitos de QoS usando ChorSim}
%% Definição de requisitos de QoS usando ChorSim
  \begin{frame}{Cenário de coreografia para a abordagem com ChorSim}
    \begin{figure}[h]
	\centering
	%\subfloat[orchestrated version\label{fig:commDiaOrch}]{\includegraphics[width=.5\linewidth]{figures/orchestration}}
	%\subfloat[choreographed version\label{fig:commDiaChor}]{\includegraphics[width=.5\linewidth]{figures/choreography}}
	\includegraphics[width=.8\linewidth]{figures/CDN_choreography-scenario.png}
	\caption{Coreografia de serviços da aplicação de CDN}
	\label{figure:scenario}
    \end{figure}
  \end{frame}


  \begin{frame}{Configuração das simulações em ChorSim (I)}  
      
      \begin{block}{Objetivo} 
      Analisar o comportamento do \textbf{tempo de resposta total} do serviço composto $WS_1$ em função do
      \textbf{tamanho da resposta} de $WS_1$ e diferentes valores de \textbf{largura de banda}.
      \end{block}

      \pause 

    \only<2>{
      \begin{itemize}
	\item \textbf{Cenário 1:}   Larguras de banda uniformes.
	\item \textbf{Cenário 2:}   Larguras de banda variáveis.
      \end{itemize}
     }
    
  \end{frame}

  \begin{frame}{Configuração das simulações em ChorSim (II)}  
	\textbf{Cenário 2:}
  	\begin{figure}[!h]
	    \centering
	    %\subfloat[orchestrated version\label{fig:commDiaOrch}]{\includegraphics[width=.5\linewidth]{figures/orchestration}}
	    %\subfloat[choreographed version\label{fig:commDiaChor}]{\includegraphics[width=.5\linewidth]{figures/choreography}}
	    \includegraphics[width=1.0\linewidth]{figures/failure_model.png}
	    \caption{Cenário 2:Largura de banda efetiva devido à degradação da largura de banda referencial em um período de $100$ segundos.}
    %       \caption{Capacidade variável da Largura de Banda  referencial em um período de tempo de $100$ segundos}
	    \label{figure:failure_model}
	\end{figure}
  \end{frame}  

  \begin{frame}{Configuração das simulações em ChorSim (III)}  
      \begin{table}[!h]
	    \centering
      {\footnotesize
	    \caption{Configuração de valores dos atributos de QoS nas requisições}
	    \label{table:simulation_configuration_responses}
	  \begin{tabular}{|l|c|c|c|c|}
		%%\hline
			    %%& \multicolumn{2}{c|}{ Average number of tokens } & \multicolumn{2}{c|}{ 95\% Confidence interval  (+/-) }\\
		\hline
		\textbf{Requisições}           &  Largura de banda     &   Tamanho da requisição    &  Latência       &  \# requisições	  \\
		\hline
		$Cliente$ a $WS_1$    &    $1$Mbps	              &      $1.95$MB        &   $0.002s$      &     $1$ a $10$     \\
		$WS_1$ a $WS_3$       &    $1$Mbps	              &      $5.47$MB        &   $0.002s$      &     $1$ a $10$     \\
		$WS_3$ a $WS_5$       &    $1$Mbps                  &      $5.47$MB        &   $0.002s$      &       $1$ a $10$     \\
		\hline
		\end{tabular}
      }
      \end{table}

      \pause
      \begin{table}[!h]
	    \centering
      {\footnotesize
	    \caption{Configuração de valores dos atributos de QoS nas respostas}
	    \label{table:simulation_configuration_requests}
	  \begin{tabular}{|l|c|c|c|c|}
		%%\hline
			    %%& \multicolumn{2}{c|}{ Average number of tokens } & \multicolumn{2}{c|}{ 95\% Confidence interval  (+/-) }\\
		\hline
		\textbf{Respostas}           &  Largura de banda           &   Tamanho de resposta    &  Latência         &   \emph{timeout}	  \\
		\hline
		$WS_1$ a $Cliente$  &    $1$Mbps a $16$Mbps	        &    $1$KB a  $100$MB                   &   $0.002s$      &   $1000s$     \\
		$WS_3$ a $WS_1$     &    $20$Mbps	                &      $8$MB                   &   $0.002s$      &   $1000s$     \\
		$WS_5$ a $WS_3$     &    $40$Mbps                 &      $200$MB                 &   $0.002s$      &   $1000s$     \\
		\hline
		\end{tabular}
      }
      \end{table}
  \end{frame}

  
  \begin{frame}{Resultados das simulações usando ChorSim}
    \only<1>{
      \framesubtitle{Cenário 1}    
      \begin{figure}[H]
	  \centering
	  %\subfloat[orchestrated version\label{fig:commDiaOrch}]{\includegraphics[width=.5\linewidth]{figures/orchestration}}
	  %\subfloat[choreographed version\label{fig:commDiaChor}]{\includegraphics[width=.5\linewidth]{figures/choreography}}
	  \includegraphics[width=1.0\linewidth]{figures/results1-normal.png}
	  \caption{\textbf{Cenário 1:} Tempo médio de resposta total da coreografia em função do tamanho de resposta do serviço $WS_1$ com
	  larguras de banda de $1Mbps$ até	$16Mbps$}
	  \label{figure:results}
      \end{figure}
    }
    \only<2>{
      \framesubtitle{Cenário 2}    
      \begin{figure}[H]
	  \centering
	  %\subfloat[orchestrated version\label{fig:commDiaOrch}]{\includegraphics[width=.5\linewidth]{figures/orchestration}}
	  %\subfloat[choreographed version\label{fig:commDiaChor}]{\includegraphics[width=.5\linewidth]{figures/choreography}}
	  \includegraphics[width=1.0\linewidth]{figures/results1-failure.png}
	  \caption{\textbf{Cenário 2:} Tempo médio de resposta total da coreografia em função do tamanho de resposta do serviço $WS_1$.
		    A largura de banda varia de $1Mbps$ até $16Mbps$}
	  \label{figure:results_failure}
      \end{figure}
    }
  \end{frame}  


%%%
\subsection{Estabelecimento do contrato de QoS}

  \begin{frame}{Estabelecimento do contrato probabilístico (I)}
      \begin{table}[!h]
	    \centering
      {\footnotesize
	    \caption{Configuração das taxas de degradação dos serviços  para obter o contrato do tempo de resposta para o serviço composto $WS_1$ }
	    \label{table:contract_configurations}
	  \begin{tabular}{ccc}
		%%\hline
			    %%& \multicolumn{2}{c|}{ Average number of tokens } & \multicolumn{2}{c|}{ 95\% Confidence interval  (+/-) }\\
		\toprule
		\textbf{Serviço}    &   \textbf{Distribuição}      &     \textbf{Taxa de degradação ($\lambda$)}      \\
		\otoprule
		$WS_1$     &    Exponencial      &      1/10000      \\
		$WS_3$     &    Exponencial      &      1/10000      \\
		$WS_5$     &    Exponencial      &      1/10000      \\
		\bottomrule
		\end{tabular}
      }
      \end{table}
  \end{frame}
  

  \begin{frame}{Estabelecimento do contrato probabilístico (II)}
    \begin{figure}[H]
	\centering
	\includegraphics[width=1.0\linewidth]{figures/contract.png}
	\caption{Distribuição de probabilidade empírica (ECDF) do contrato do serviço $WS_1$ com base nos tempos de resposta  }
	\label{figure:contract}
    \end{figure}
  \end{frame}


%%%
\subsection{Monitoramento}

  \begin{frame}{Configuração do Monitoramento}
    \begin{equation}
      \textcolor{Blue}{
	  verify(X_{monitoring}) = 	
	  \begin{cases}
	  true,  &  \text{ se } p^{+}>= 0.05 \wedge D^{+} < 0.15 \\
	  false, &  \text{ de outra maneira }
	  \end{cases}
      }
    \end{equation}
  
    \pause
    \begin{itemize}
      \item \textcolor{Blue}{$X_{monitoring}$}: Conjunto de amostras dos tempos de resposta do serviço composto \textcolor{Blue}{$WS_1$}. 
    \pause
      \item Um monitoramento sequencial e  \textit{on-line} precisa  computar  \textcolor{Blue}{$N$} amostras que que se sobrepõem. 
      \item Conjunto de janelas de  amostras a monitorar: 
	    \begin{itemize}
	     \item [] \textcolor{Blue}{$\{1, \dots , N\}, \{p, \dots , p + N\}, \dots , \{mp, \dots ,mp + N\} \dots$ }
	     \item [] \textcolor{Blue}{$p <=N $} e \textcolor{Blue}{$m=1, 2,\dots \quad$ }
	    \end{itemize}
     \pause
      \item Configuração: \textcolor{Blue}{$N = 100$}, e o desvio \textcolor{Blue}{$p=1$}.
    \end{itemize}

  \end{frame}


%\colorbox{green!25}{}
%\colorbox{red!25}{$3.10\%$}
  \begin{frame}{Cenários para o monitoramento}
    \begin{center}
      \begin{table}[h]
	  \centering
	\caption{ Estabelecimento das taxas de degradação de tempo de processamento dos serviços para os cenários.  }
	\begin{center}
	  \begin{tabular}{ cccc}
	    %\hline
	    % after \\: \hline or \cline{col1-col2} \cline{col3-col4} ...
	    \toprule[1pt]
			&   \multicolumn{3}{c}{ \textbf{Taxa de degradação ($\lambda$)} } \\
	    %\cline{1-4}
	    %\cmidrule{1-4}
	    \textbf{Cenário}    &    $WS_1$	  &	  $WS_3$ 	& 	 $WS_5$ \\
	    %\otoprule
	    \midrule[1pt]
	    %\cmidrule[1pt](r){1-1} \cmidrule[1pt](l){2-4}
	    %\rowcolor{yellow!35}
	    %\rowcolor{red!25}
	    %\midrule
	    Cenário 1 &		\cellcolor{red!25} 1/13000  	 & 	\cellcolor{red!25} 1/12500		&	\cellcolor{red!25} 1/11500	\\
	    %\rowcolor{red!25}
	    Cenário 2 &		\cellcolor{red!25} 1/12000  	 & 	\cellcolor{red!25} 1/11000		&	\cellcolor{red!25} 1/12000	\\
	    \midrule
	    %\rowcolor{green!25}
	    Cenário 3 &		\cellcolor{green!25} 1/10500  	 & 	\cellcolor{green!25} 1/10000		&	\cellcolor{green!25} 1/10500	\\
	    %\rowcolor{green!25}
	    Cenário 4 &		\cellcolor{green!25} 1/9000  	 & 	\cellcolor{green!25} 1/10000		&	\cellcolor{green!25} 1/9000	\\
	    %\hline
	    \midrule[1pt]
	    %\cmidrule[1pt](r){1-1} \cmidrule[1pt](l){2-4}
	    \textbf{Contrato}  &	\cellcolor{yellow!49} 1/10000		&	\cellcolor{yellow!49} 1/10000		&	\cellcolor{yellow!49} 1/10000 \\
	    \bottomrule[1pt]
      
	  \end{tabular}
	  \label{table:scenarios_rates}
	  \end{center}
      \end{table}
    \end{center}
  \end{frame}

  \begin{frame}{Resultados da detecção de violações de SLA (I)} 
  	\only<1>{
	  \begin{figure}[H]
	      \centering
	      \includegraphics[width=1.0\linewidth]{figures/KS-Comparison.png}
	      \caption{ Comparação do ECDF do contrato com o ECDF do cenário 1 e 4. O tamanho das amostras dos cenários é 100. }
	  \end{figure}
	}
  	\only<2>{
	  \begin{figure}[H]
	      \centering
	      \includegraphics[width=1.0\linewidth]{figures/KS-Comparison1.png}
	      \caption{ Comparação do ECDF do contrato com o ECDF do cenário 1}
	  \end{figure}
	}
  	\only<3>{
	  \begin{figure}[H]
	      \centering
	      \includegraphics[width=1.0\linewidth]{figures/KS-Comparison2.png}
	      \caption{ Comparação do ECDF do contrato com o ECDF do cenário 4}
	  \end{figure}
	}
  \end{frame}
 
  \begin{frame}{Resultados da detecção de violações de SLA (II) - \textit{Online}}  
	\framesubtitle{Cenário \textbf{1}}    
	\only<1>{
	  \begin{figure}[H]
	      \centering
	      \includegraphics[width=0.6\linewidth]{figures/detection1-pie-13_125_115.png}
	      \caption{ Violações de SLA para o cenário 1.  }
	      \label{figure:detection1-13_125_115}
	  \end{figure}
	}
	\only<2>{
	  \begin{figure}[H]
	      \centering
	      \includegraphics[width=1.0\linewidth]{figures/detection_13_125_115.png}
	      \caption{ Monitoramento e detecção de violações de SLA para o cenário 1.  }
	      \label{figure:detection1-13_125_115}
	  \end{figure}
	}
    \end{frame}

\begin{comment}
  \begin{frame}{Resultados da detecção de violações de SLA }  
	\framesubtitle{Cenário \textbf{2}}    
	\only<1>{
	  \begin{figure}[H]
	      \centering
	    \includegraphics[width=0.6\linewidth]{figures/detection2-pie-12_11_12.png}
	      \caption{ Violações de SLA para o cenário 2.  }
	      \label{figure:detection1-12_11_12}
	  \end{figure}
	}
	\only<2>{
	  \begin{figure}[H]
	      \centering
	      \includegraphics[width=1.0\linewidth]{figures/detection_12_11_12.png}
	      \caption{ Monitoramento e detecção de violações de SLA para o cenário 2.  }
	      \label{figure:detection1-12_11_12}
	  \end{figure}
	}
    \end{frame}
\end{comment}


  \begin{frame}{Resultados da detecção de violações de SLA }  
	\framesubtitle{Cenário \textbf{3}}    
	\only<1>{
	  \begin{figure}[H]
	      \centering
	  \includegraphics[width=0.5\linewidth]{figures/detection4-pie-105_10_105.png}
	      \caption{ Violações de SLA para o cenário 3.  }
	      \label{figure:detection1-105_10_105}
	  \end{figure}
	}
	\only<2>{
	  \begin{figure}[H]
	      \centering
	  \includegraphics[width=1.0\linewidth]{figures/detection_105_10_105.png}
	      \caption{ Monitoramento e detecção de violações de SLA para o cenário 3.  }
	      \label{figure:detection1-105_10_105}
	  \end{figure}
	}
    \end{frame}

  \begin{frame}{Resultados da detecção de violações de SLA }  
	\framesubtitle{Cenário \textbf{4}}    
	\only<1>{
	  \begin{figure}[H]
	      \centering
	  \includegraphics[width=0.6\linewidth]{figures/detection5-pie-9_10_9.png}
	      \caption{ Violações de SLA para o cenário 4.  }
	      \label{figure:detection1-13_125_115}
	  \end{figure}
	}
	\only<2>{
	  \begin{figure}[H]
	      \centering
	      \includegraphics[width=1.0\linewidth]{figures/detection_9_10_9.png}
	      \caption{ Monitoramento e detecção de violações de SLA para o cenário 4.  }
	      \label{figure:detection1-13_125_115}
	  \end{figure}
	}
    \end{frame}
    


\section{Conclusões e Trabalhos Futuros}
  \begin{frame}{Conclusões (I)}
      \begin{itemize}
	\item Foram propostas  abordagens e técnicas para definição de requisitos de QoS, estabelecimento de contratos probabilísticos
	e monitoramento com a detecção de violações de contrato em coreografias de serviços Web.
	\item O modelo de interação em BPMN2 foi utilizado.
	\item Foi construído um simulador para suportar \textit{enactment} de coreografias de serviços Web com suporte de QoS.
	\item Na definição de requisitos, foi proposta uma metodologia usando mapeamento de coreografia para GSPN. Além disso, 
	apresentou-se avaliações de desempenho usando o ChorSim também.
      \end{itemize}

  \end{frame}

  \begin{frame}{Conclusões (II)}
      \begin{itemize}
	\item Apresentou-se uma abordagem para estabelecer contratos probabilísticos usando ChorSim.
	\item O monitor foi desenvolvido de modo a suportar um monitoramento \textit{online}.
	\item Foi proposta uma técnica de dominância estocástica para detectar violações de contratos probabilísticos. 
	  A técnica está baseada no teste de \textit{Kolmogorov-Smirnov} de apenas um lado.
	\item Os resultados mostram que na regra para detectar violações o \textit{p-value} pode ser suficiente.
      \end{itemize}
  \end{frame}

  \begin{frame}{Trabalhos Futuros}
      \begin{itemize}
	\item Suporte de mais atributos de QoS de desempenho e outros como confiabilidade, segurança, entre outros, em todas as etapas.
	\item Suporte de mais elementos de coreografia de serviços do BPMN2 em todas as etapas.
%%suporte completo de coreografia de processos, automatização do mapeamento de uma especificação de coreografia em BPMN2 para o enactment no ChorSim
	\item Pesquisa em  substituição dinâmica, adaptação, reconfiguração, autocura, entre outros,  baseada em QoS para coreografias de serviços Web.
	\item Suporte de definição probabilística de atributos de QoS como largura de banda e latência de rede no ChorSim.
%%chorsim, monitoramento
	\item Avaliações e validações das simulações com implementações reais de coreografias.
	\item Levar em consideração exemplos ou cenários reais e/ou complexos.
      \end{itemize}
  \end{frame}
  


  \begin{frame}{Publicações}
      \begin{thebibliography}{1}
        \beamertemplatearticlebibitems
	\bibitem[CAMAD,2012]{APDIAZ2012}
	    A. Diaz, D. Batista
	  \newblock {\em A methodology to define QoS and SLA requirements in service choreographies}.
	  \newblock  IEEE 17th International Workshop on Computer Aided Modeling and Design of Communication Links and Networks (CAMAD'12). 2012:201-205
      \end{thebibliography}
  \end{frame}




%----------------- Bibliography ---------------------------------------
%\bibliographystyle{alpha-ime}% citação bibliográfica alpha
%\begin{frame}[allowframebreaks]{Referências}
 %   \begin{thebibliography}{10}
        %\beamertemplatebookbibitems


         %\bibitem[Label1, 2010]{key1} Author's name (1987)
         %\newblock Title of the paper.
         %\newblock \emph{Journal Name} 55(4), 765 -- 799.
%         \beamertemplatearticlebibitems
%         \bibitem[Rosario et al., 2008]{Rosario2008}
%            Rosario S, Benveniste A, Haar S, Jard C.
%            \newblock {\em Probabilistic QoS and Soft Contracts for Transaction-Based Web Services Orchestrations}.
%            \newblock IEEE Transactions on Services Computing. 2008;1(4):187-200.

%          \beamertemplatearticlebibitems
%          \bibitem[Rosenberg et al.,2006]{Rosenberg2006}
%            Rosenberg F, Platzer C, Dustdar S.
%            \newblock {\em Bootstrapping Performance and Dependability Attributes of Web Services}.
%            \newblock IEEE International Conference on Web Services (ICWS’06). 2006:205-212.


%          \beamertemplatearticlebibitems
%          \bibitem[UlHaq et al.,2010]{UlHaq2010}
%            Ul Haq I, Paschke A, Schikuta E, Boley H.
%            \newblock {\em Rule-based validation of SLA choreographies}.
%            \newblock The Journal of Supercomputing. 2010.

%          \beamertemplatearticlebibitems
%          \bibitem[Rosenberg et al.,2007]{Rosenberg2007}
%            Rosenberg F, Enzi C, Michlmayr A, Platzer C, Dustdar S.
%            \newblock {\em Integrating Quality of Service Aspects in Top-Down Business Process Development Using WS-CDL and %            WS-BPEL}.
%            \newblock 11th IEEE International Enterprise Distributed Object Computing Conference (EDOC 2007). 2007:15-15.

%          \beamertemplatearticlebibitems
%          \bibitem[Michlmayr et al.,2009]{Michlmayr2009}
%            Michlmayr A, Rosenberg F, Leitner P, Dustdar S.
%            \newblock {\em Comprehensive QoS monitoring of Web services and event-based SLA violation detection}.
%            \newblock Proceedings of the 4th International Workshop on Middleware for Service Oriented Computing - MWSOC  ’09. %2009:1-6.



 %\end{thebibliography}


%\bibliographystyle{apalike}% citação bibliográfica alpha
%\def\newblock{}
 %       \bibliography{slides}
 %   \end{thebibliography}
%\end{frame}




%--------Envolvimento --------------
    \begin{frame}{Envolvimento}
      	\begin{figure}[!h]
	    \centering
	    \includegraphics[scale=0.6]{CHOReOSProject.png}
	\end{figure}

	\begin{figure}[!h]
	    \centering
	    \includegraphics[width=0.28\textwidth]{hp_logo_2.png}
	\end{figure}
	
	\begin{figure}[!h]
	    \centering
	    \includegraphics[width=0.2\textwidth]{logo_cnpq.jpg}
	\end{figure}
	
    \end{frame}

%----------- Obrigado -------------
 \begin{frame}
              %\begin{figure}[!h]
               %   \centering
               %   \includegraphics[width=0.7\textwidth]{./figures/misti2.jpg}
              %\end{figure}	
%	\fbox{\textbf{\Large{Muito Obrigado}}}
	\makebox[\textwidth]{
	  \color{Blue}{
	    \textbf{\huge{Obrigado!}}	
	    %\end{center}
	   }
    }
 \end{frame}



\end{document} 
